\documentclass[xcolor=dvipsnames]{beamer}

\usetheme{Madrid}
\useoutertheme{miniframes} % Alternatively: miniframes, infolines, split
\useinnertheme{circles}

\setbeamerfont{frametitle}{size=\footnotesize}

\setbeamercolor{section in toc}{fg=black,bg=black}
\setbeamercolor{alerted text}{fg=black}
\setbeamercolor*{palette primary}{fg=white,bg=gray} % cadre autour du titre + rectangle bas droite.
\setbeamercolor*{palette secondary}{fg=yellow,bg=yellow}
\setbeamercolor*{palette tertiary}{bg=black,fg=gray!10!white} % carré du milieu + bande en header. 
\setbeamercolor*{palette quaternary}{fg=yellow,bg=yellow}
\setbeamercolor*{sidebar}{fg=black,bg=gray!15!white}
%\setbeamercolor*{titlelike}{parent=palette primary}
\setbeamercolor{titlelike}{parent=palette primary,fg=black}
\setbeamercolor{frametitle}{bg=white}
\setbeamercolor{frametitle right}{bg=gray!60!white}
\setbeamercolor*{separation line}{}
\setbeamercolor*{fine separation line}{}
% Override palette coloring with secondary
\setbeamercolor{subsection in head/foot}{bg=gray,fg=white}
\setbeamercolor{local structure}{fg=black} % changer la couleur des enumerate et itemize.

\title[Analyse d'articles scientifiques avec R]{Analyse textuelle d’articles scientifiques évaluant l’impact des vers de terre sur l’environnement}
\subtitle{Une approche \textit{tidy} pour l'analyse textuelle d'articles scientifiques avec \includegraphics[scale=0.04]{R_logo.svg.png}}
\author[Stage M1 BIMS]{Antoine MALET - Stage M1, Parcours BIMS}
\institute[UMR MIA Paris-Saclay]{Campus Agro Paris-Saclay - Unité Mathématiques et Informatique Appliqués}
\date{03/07/2024}

\begin{document}

\setbeamertemplate{headline}{}

	\begin{frame}
		\titlepage
	\end{frame}
	
\setbeamertemplate{headline}[miniframes theme]

	\section*{Introduction}
	\subsection*{Contexte scientifique / Objectifs} % PBTQ, à remettre dans chaque ss.

	\begin{frame}
		\frametitle{\underline{Contexte scientifique:}}
		Bla bla 
		\begin{columns}
			\begin{column}{0.5\textwidth} % permet diviser la frame en deux colonnes, chacune occupant 50% de l'espace total disponible.
				\begin{enumerate}
					\item List item 1
					\item List item 2
				\end{enumerate}
			\end{column}
			\begin{column}{0.5\textwidth}
				\begin{itemize}
					\item List item 1
					\item List item 2
				\end{itemize}
			\end{column}
		\end{columns}
		\vspace{\baselineskip}
		Here is some rambling text
	\end{frame}

	\begin{frame}
		\frametitle{\underline{Objectifs:}}
		Bla bla 
		\begin{columns}
			\begin{column}{0.5\textwidth} % permet diviser la frame en deux colonnes, chacune occupant 50% de l'espace total disponible.
				\begin{enumerate}
					\item List item 1
					\item List item 2
				\end{enumerate}
			\end{column}
			\begin{column}{0.5\textwidth}
				\begin{itemize}
					\item List item 1
					\item List item 2
				\end{itemize}
			\end{column}
		\end{columns}
		\vspace{\baselineskip}
		Here is some rambling text
	\end{frame}

	\section*{Données et scripts}
	\subsection*{Base de données / Scripts} % PBTQ, à remettre dans chaque ss.
	\begin{frame}
		\frametitle{\underline{Base de données:}}
		Bla bla 
		\begin{columns}
			\begin{column}{0.5\textwidth} % permet diviser la frame en deux colonnes, chacune occupant 50% de l'espace total disponible.
				\begin{enumerate}
					\item List item 1
					\item List item 2
				\end{enumerate}
			\end{column}
			\begin{column}{0.5\textwidth}
				\begin{itemize}
					\item List item 1
					\item List item 2
				\end{itemize}
			\end{column}
		\end{columns}
		\vspace{\baselineskip}
		Here is some rambling text
	\end{frame}

	\begin{frame}
		\frametitle{\underline{Scripts Python et R:}}
		Bla bla 
		\begin{columns}
			\begin{column}{0.5\textwidth} % permet diviser la frame en deux colonnes, chacune occupant 50% de l'espace total disponible.
				\begin{enumerate}
					\item List item 1
					\item List item 2
				\end{enumerate}
			\end{column}
			\begin{column}{0.5\textwidth}
				\begin{itemize}
					\item List item 1
					\item List item 2
				\end{itemize}
			\end{column}
		\end{columns}
		\vspace{\baselineskip}
		Here is some rambling text
	\end{frame}

	\section*{Calculs de fréquence}
	\subsection*{Courbe de Zipf / Fréquences brutes / Comparaison de fréquences / Approche TF-IDF}

	\begin{frame}
		\frametitle{\underline{Courbe de Zipf:}}
		Bla bla 
		\begin{columns}
			\begin{column}{0.5\textwidth} % permet diviser la frame en deux colonnes, chacune occupant 50% de l'espace total disponible.
				\begin{enumerate}
					\item List item 1
					\item List item 2
				\end{enumerate}
			\end{column}
			\begin{column}{0.5\textwidth}
				\begin{itemize}
					\item List item 1
					\item List item 2
				\end{itemize}
			\end{column}
		\end{columns}
		\vspace{\baselineskip}
		Here is some rambling text
	\end{frame}

	\begin{frame}
		\frametitle{\underline{Fréquences brutes:}}
		Bla bla 
		\begin{columns}
			\begin{column}{0.5\textwidth} % permet diviser la frame en deux colonnes, chacune occupant 50% de l'espace total disponible.
				\begin{enumerate}
					\item List item 1
					\item List item 2
				\end{enumerate}
			\end{column}
			\begin{column}{0.5\textwidth}
				\begin{itemize}
					\item List item 1
					\item List item 2
				\end{itemize}
			\end{column}
		\end{columns}
		\vspace{\baselineskip}
		Here is some rambling text
	\end{frame}

	\begin{frame}
		\frametitle{\underline{Comparaison de fréquences:}}
		Bla bla 
		\begin{columns}
			\begin{column}{0.5\textwidth} % permet diviser la frame en deux colonnes, chacune occupant 50% de l'espace total disponible.
				\begin{enumerate}
					\item List item 1
					\item List item 2
				\end{enumerate}
			\end{column}
			\begin{column}{0.5\textwidth}
				\begin{itemize}
					\item List item 1
					\item List item 2
				\end{itemize}
			\end{column}
		\end{columns}
		\vspace{\baselineskip}
		Here is some rambling text
	\end{frame}

	\begin{frame}
		\frametitle{\underline{Approche TF-IDF:}}
		Bla bla 
		\begin{columns}
			\begin{column}{0.5\textwidth} % permet diviser la frame en deux colonnes, chacune occupant 50% de l'espace total disponible.
				\begin{enumerate}
					\item List item 1
					\item List item 2
				\end{enumerate}
			\end{column}
			\begin{column}{0.5\textwidth}
				\begin{itemize}
					\item List item 1
					\item List item 2
				\end{itemize}
			\end{column}
		\end{columns}
		\vspace{\baselineskip}
		Here is some rambling text
	\end{frame}

	\section*{Analyse de bigrammes}
	\subsection*{TF-IDF sur les bigrammes / Réseaux de bigrammes}

	\begin{frame}
		\frametitle{\underline{TF-IDF sur les bigrammes:}}
		Bla bla 
		\begin{columns}
			\begin{column}{0.5\textwidth} % permet diviser la frame en deux colonnes, chacune occupant 50% de l'espace total disponible.
				\begin{enumerate}
					\item List item 1
					\item List item 2
				\end{enumerate}
			\end{column}
			\begin{column}{0.5\textwidth}
				\begin{itemize}
					\item List item 1
					\item List item 2
				\end{itemize}
			\end{column}
		\end{columns}
		\vspace{\baselineskip}
		Here is some rambling text
	\end{frame}

	\begin{frame}
		\frametitle{\underline{Réseaux de bigrammes:}}
		Bla bla 
		\begin{columns}
			\begin{column}{0.5\textwidth} % permet diviser la frame en deux colonnes, chacune occupant 50% de l'espace total disponible.
				\begin{enumerate}
					\item List item 1
					\item List item 2
				\end{enumerate}
			\end{column}
			\begin{column}{0.5\textwidth}
				\begin{itemize}
					\item List item 1
					\item List item 2
				\end{itemize}
			\end{column}
		\end{columns}
		\vspace{\baselineskip}
		Here is some rambling text
	\end{frame}

	\section*{Analyse de sentiment}
	\subsection*{Wordclouds / Mots simples / Bigrammes}

	\begin{frame}
		\frametitle{\underline{Wordclouds:}}
		Bla bla 
		\begin{columns}
			\begin{column}{0.5\textwidth} % permet diviser la frame en deux colonnes, chacune occupant 50% de l'espace total disponible.
				\begin{enumerate}
					\item List item 1
					\item List item 2
				\end{enumerate}
			\end{column}
			\begin{column}{0.5\textwidth}
				\begin{itemize}
					\item List item 1
					\item List item 2
				\end{itemize}
			\end{column}
		\end{columns}
		\vspace{\baselineskip}
		Here is some rambling text
	\end{frame}

	\begin{frame}
		\frametitle{\underline{Mots simples:}}
		Bla bla 
		\begin{columns}
			\begin{column}{0.5\textwidth} % permet diviser la frame en deux colonnes, chacune occupant 50% de l'espace total disponible.
				\begin{enumerate}
					\item List item 1
					\item List item 2
				\end{enumerate}
			\end{column}
			\begin{column}{0.5\textwidth}
				\begin{itemize}
					\item List item 1
					\item List item 2
				\end{itemize}
			\end{column}
		\end{columns}
		\vspace{\baselineskip}
		Here is some rambling text
	\end{frame}

	\begin{frame}
		\frametitle{\underline{Bigrammes:}}
		Bla bla 
		\begin{columns}
			\begin{column}{0.5\textwidth} % permet diviser la frame en deux colonnes, chacune occupant 50% de l'espace total disponible.
				\begin{enumerate}
					\item List item 1
					\item List item 2
				\end{enumerate}
			\end{column}
			\begin{column}{0.5\textwidth}
				\begin{itemize}
					\item List item 1
					\item List item 2
				\end{itemize}
			\end{column}
		\end{columns}
		\vspace{\baselineskip}
		Here is some rambling text
	\end{frame}

	\section*{Conclusion}
	\subsection*{Rappel des résultats et perspectives}

	\begin{frame}
		Bla bla 
		\begin{columns}
			\begin{column}{0.5\textwidth} % permet diviser la frame en deux colonnes, chacune occupant 50% de l'espace total disponible.
				\begin{enumerate}
					\item List item 1
					\item List item 2
				\end{enumerate}
			\end{column}
			\begin{column}{0.5\textwidth}
				\begin{itemize}
					\item List item 1
					\item List item 2
				\end{itemize}
			\end{column}
		\end{columns}
		\vspace{\baselineskip}
		Here is some rambling text
	\end{frame}

\end{document}

















% 15 DIAPOS EN M1 BIMS.

% MANUEL:

% FAIRE UNE FRAME COUPÉE EN 2 AU MILIEU:
%  	\begin{frame}
% 	Bla bla 
% 	\begin{columns}
% 		\begin{column}{0.5\textwidth} % permet diviser la frame en deux colonnes, chacune occupant 50% de l'espace total disponible.
% 			\begin{enumerate}
% 				\item List item 1
% 				\item List item 2
% 			\end{enumerate}
% 		\end{column}
% 		\begin{column}{0.5\textwidth}
% 			\begin{itemize}
% 				\item List item 1
% 				\item List item 2
% 			\end{itemize}
% 		\end{column}
% 	\end{columns}
% 	\vspace{\baselineskip}
	
% 	Here is some rambling text

% \end{frame}